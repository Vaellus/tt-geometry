\usepackage{amssymb,xcolor,graphicx,amsthm,multicol,mathtools,amsfonts,parskip,thmtools,mathrsfs}
\usepackage{tikz-cd}
\usepackage[shortlabels]{enumitem}
\usepackage{hyperref}
\usepackage[all, knot]{xy}
\xyoption{all}
\xyoption{arc}

\usepackage[cal=cm, scr=rsfs, bb=ams]{mathalpha}

\DeclareSymbolFont{PazoBB}{U}{fplmbb}{m}{n}
\DeclareMathSymbol{\unity}{\mathord}{PazoBB}{"31}

% \usepackage{fontspec}
% \usepackage{unicode-math}

% \setmathfont{LatinModern-Math.otf}

% \DeclareMathAlphabet{\mathcal}{OMS}{cmsy}{m}{n}
% \let\mathbb\relax % remove the definition by unicode-math
% \DeclareMathAlphabet{\mathbb}{U}{msb}{m}{n}

\parindent = 10pt
\oddsidemargin=0in 
\evensidemargin=0in 
\topmargin=-.5in 
\textwidth=6in 
\textheight=9in

\newcommand{\ges}{\geqslant}
\newcommand{\les}{\leqslant}
\newcommand{\vf}{\varphi}
\newcommand{\lra}{\longrightarrow}
\newcommand{\hra}{\hookrightarrow}
\newcommand{\lms}{\longmapsto}
\newcommand{\Aut}{\operatorname{Aut}}
\newcommand{\Ker}{\operatorname{Ker}}
\newcommand{\Gl}{\operatorname{GL}}
\newcommand{\Sl}{\operatorname{SL}}
\newcommand{\fc}{{\mathfrak c}}
\newcommand{\fp}{{\mathfrak p}}
\newcommand{\fm}{{\mathfrak m}}
\newcommand{\fn}{{\mathfrak n}}
\newcommand{\fq}{{\mathfrak q}}\DeclareMathOperator{\trdeg}{trdeg}

\newcommand\cechit[1]{\smash{\check{#1}}\vphantom{#1}}
\DeclareMathOperator\chh{\cechit{\hh}}
\DeclareMathOperator{\cone}{cone}
\DeclareMathOperator{\id}{id}
\DeclareMathOperator{\pdim}{pdim}
\DeclareMathOperator{\pd}{pdim}
\DeclareMathOperator{\edim}{edim}
\DeclareMathOperator{\gldim}{gl-dim}
\DeclareMathOperator{\Tor}{Tor}
\DeclareMathOperator{\Ext}{Ext}
\DeclareMathOperator{\mSpec}{mSpec}
\DeclareMathOperator{\Spec}{Spec}
\DeclareMathOperator{\Ass}{Ass}
\DeclareMathOperator{\ann}{Ann}
\DeclareMathOperator{\Ann}{Ann}
\DeclareMathOperator{\Min}{Min}
\DeclareMathOperator{\Supp}{Supp}
\DeclareMathOperator{\hgt}{height}
\DeclareMathOperator{\height}{height}
\DeclareMathOperator{\Der}{Der}
\DeclareMathOperator{\Hom}{Hom}
\DeclareMathOperator{\sing}{Sing}
\newcommand{\Ob}{\mathrm{Ob}}
\newcommand{\Set}{\mathbf{Set}}
\newcommand{\susp}{\mathsf{\Sigma}}
\DeclareMathOperator{\PO}{\mathbf{PO}}
\DeclareMathOperator{\coker}{coker}
\DeclareMathOperator{\sign}{Sign}
\DeclareMathOperator{\im}{Im}
\DeclareMathOperator{\Span}{Span}
\DeclareMathOperator{\rank}{rank}
\DeclareMathOperator{\hh}{H}
\DeclareMathOperator{\grade}{grade}
\DeclareMathOperator{\depth}{depth}
\DeclareMathOperator{\soc}{Soc}
\DeclareMathOperator{\TQ}{TQ}
\DeclareMathOperator{\gr}{gr}
\DeclareMathOperator{\ord}{ord}

\DeclareMathOperator{\gl}{gl}
\DeclareMathOperator{\chr}{char}
\DeclareMathOperator{\rep}{rep}
\DeclareMathOperator{\Rep}{Rep}
\DeclareMathOperator{\Sym}{Sym}
\DeclareMathOperator{\ad}{ad}
\DeclareMathOperator{\g}{\mathfrak{g}}

\newcommand{\C}{\mathbb C}
\newcommand{\PP}{\mathbb P}
\newcommand{\F}{\mathbb F}
\newcommand{\N}{\mathbb N}
\newcommand{\Q}{\mathbb Q}
\newcommand{\R}{\mathbb R}
\newcommand{\A}{\mathbb A}
\newcommand{\FF}{\mathbb F}
\newcommand{\Z}{\mathbb Z}
\newcommand{\ZZ}{\mathcal Z}
\newcommand{\I}{\mathcal I}
\newcommand{\arrow}{\overrightarrow}
\newcommand{\xra}{\xrightarrow}
\newcommand{\la}{\langle}
\newcommand{\ra}{\rangle}
\newcommand{\forany}{\forall}
\newcommand{\ind}{\raisebox{.45ex}{$\chi$}}
\newcommand{\lt}{\left}
\newcommand{\rt}{\right}


\newenvironment{soln}{\begin{proof}[Solution]\hfill } {\end{proof}}

\declaretheorem[numberwithin=section,name={Theorem}, refname = {theorem,theorems},Refname={Theorem,Theorems}]{thm}
\declaretheorem[name={Lemma},sibling=thm, refname = {lemma,lemmas},Refname={Lemma,Lemmas}]{lem}
\declaretheorem[sibling=thm,name=Corollary, refname={corollary, corollaries},Refname={Corollary,Corollaries}]{cor}
\declaretheorem[sibling=thm,name=Proposition,refname={proposition,propositions},Refname={Proposition,Propositions}]{prop}
\declaretheorem[sibling=thm,name=Definition,style=definition]{defn}
\declaretheorem[sibling=thm,name=Construction,style=definition]{const}
\declaretheorem[sibling=thm,name=Example,style=definition,refname={example,examples}, Refname={Example, Examples}]{xmpl}
\declaretheorem[sibling=thm,name=Examples,style=definition,refname={example,examples}, Refname={Example, Examples}]{xmpls}
\declaretheorem[sibling=thm,name=Exercise,style=definition,refname={exercise,exercises}, Refname={Exercise, Exercises}]{xcs}
\declaretheorem[sibling=thm,style=remark,name=Remark,refname={remark,remarks}, Refname={Remark,Remarks}]{rmk}
\declaretheorem[sibling=thm,style=remark,name=Observation,refname={obs,obsvs}, Refname={Observation,Observations}]{obs}
\declaretheorem[sibling=thm,style=definition,name=Fact,refname={fact,facts},Refname={Fact,Facts}]{fact}
\declaretheorem[style=remark,numbered=no,name=Remark]{rmk*}

\declaretheorem[name=Problem,style=definition,refname={problem,problems}, Refname={Problem,Problems}]{problem}
